\documentclass{article}
\usepackage[landscape]{geometry}
\usepackage{url}
\usepackage{multicol}
\usepackage{amsmath}
\usepackage{amsfonts}
\usepackage{tikz}
\usetikzlibrary{decorations.pathmorphing}
\usepackage{amsmath,amssymb}

\usepackage{colortbl}
\usepackage{xcolor}
\usepackage{mathtools}
\usepackage{amsmath,amssymb}
\usepackage{enumitem}
\usepackage{listings}

\title{MatPlotLib and Random Cheat Sheet}
\usepackage[brazilian]{babel}
\usepackage[utf8]{inputenc}

\advance\topmargin-.8in
\advance\textheight3in
\advance\textwidth3in
\advance\oddsidemargin-1.5in
\advance\evensidemargin-1.5in
\parindent0pt
\parskip2pt
\newcommand{\hr}{\centerline{\rule{3.5in}{1pt}}}
%\colorbox[HTML]{e4e4e4}{\makebox[\textwidth-2\fboxsep][l]{texto}
\colorlet{mygray}{black!30}
\colorlet{mygreen}{green!60!blue}
\colorlet{mymauve}{red!60!blue}

\lstset{
  basicstyle=\footnotesize\ttfamily,
  columns=fullflexible,
  breakatwhitespace=false,
  breaklines=true,
  captionpos=b,
  commentstyle=\color{mygreen},
  extendedchars=true,
  belowskip =0pt,
  keepspaces=true,
  keywordstyle=\color{blue},
  language=c++,
  numbers=none,
  numbersep=5pt,
  numberstyle=\tiny\color{blue},
  rulecolor=\color{black},
  showspaces=false,
  showtabs=false,
  stringstyle=\color{mymauve},
  tabsize=3
}
\begin{document}

\begin{center}
    \noindent\footnotesize\textbf{Handbook USM}
\end{center}
\begin{multicols*}{3}

\tikzstyle{mybox} = [draw=black, fill=white, very thick,
    rectangle, rounded corners, inner sep=10pt, inner ysep=3.5pt]
\tikzstyle{fancytitle} =[fill=black, text=white, font=\bfseries]

\begin{tikzpicture}
\node [mybox] (box){
\begin{minipage}{0.3\textwidth}\begin{lstlisting}
struct UnionFind {
    vector<int> p, r;
    UnionFind(int n) {
        r.assign(n+1, 0);
        p.assign(n+1, 0);
        for(int i=1; i<=n; i++) p[i] = i;
    }
    int findSet(int i) {
        return (p[i] == i)? i:(p[i] = findSet(p[i]));
    }
    bool isSameSet(int i, int j) {
        return findSet(i) == findSet(j);
    }
    void unionSet(int i, int j) {
        if (!isSameSet(i, j)) {
            int x = findSet(i), y = findSet(j);
            if (r[x] > r[y]) p[y] = x;
            else {
                p[x] = y;
                if (r[x] == r[y]) r[y]++;
            }
        }
    }
};\end{lstlisting}
\end{minipage}
};\node[fancytitle, right=10pt] at (box.north west) {\footnotesize Union Find};
\end{tikzpicture}
\begin{tikzpicture}
\node [mybox] (box){
\begin{minipage}{0.3\textwidth}\begin{lstlisting}
#define gcd(a, b) __gcd(a, b)
#define lcm(a, b) gcd(a, b) ? (a / gcd(a, b) * b) : 0
const double PI = 3.1415926535897932384626433832795;
const ll PRIME_BASE = (1 << 61) - 1;\end{lstlisting}
\end{minipage}
};\node[fancytitle, right=10pt] at (box.north west) {\footnotesize Mathematics};
\end{tikzpicture}
\begin{tikzpicture}
\node [mybox] (box){
\begin{minipage}{0.3\textwidth}\begin{lstlisting}
ll binpow(ll a, ll b, ll mod) {
    a %= m;
    ll res = 1;
    while (b > 0) {
        if (b & 1)
            res = (res * a) % mod;
        a = (a * a) % mod;
        b >>= 1;
    }
    return res;
}\end{lstlisting}
\end{minipage}
};\node[fancytitle, right=10pt] at (box.north west) {\footnotesize Binary Pow};
\end{tikzpicture}
\begin{tikzpicture}
\node [mybox] (box){
\begin{minipage}{0.3\textwidth}\begin{lstlisting}
struct Fraction {
    ll numerator, denominator;
    Fraction(ll a, ll b){
        numerator = a, denominator = b;
    }
    
    Fraction simplify(Fraction f){
        ll g = gcd(f.numerator, f.denominator);
        return Fraction(f.numerator/g, f.denominator/g);
    }

    Fraction add(Fraction f){
        ll l = lcm(denominator, f.denominator);
        numerator *= (l/denominator);
        numerator += f.numerator * (l/f.denominator);
        return simplify(Fraction(numerator, l));
    }
};\end{lstlisting}
\end{minipage}
};\node[fancytitle, right=10pt] at (box.north west) {\footnotesize Fraction};
\end{tikzpicture}
\begin{tikzpicture}
\node [mybox] (box){
\begin{minipage}{0.3\textwidth}\begin{lstlisting}
class RollingHashing {
  ll p, m, ns;
  vector< ll > pows, hash;

  RollingHashing(string s) {
    // if WA then other p and other m
    // if still WA then double hashing
    // if still WA maybe is not the answer RH
    p = 31; m = 1e9 + 7;

    ns = s.size();
    pows.resize(ns + 2);
    for(int i = 1; i < ns + 2; i++) 
      pows[i] = (pows[i - 1] * p) % m;

    hash.resize(ns + 1);
    hash[0] = 0;
    for(int i = 1; i <= ns; i++) {
      ll char_to_num = S[i - 1] - 'a' + 1;  
      ll prev_hash = hash[i - 1];
      hash[i] = ((char_to_num * pows[i - 1]) % m + prev_hash) % m; 
    }
  }

  ll compute_hashing(ll i, ll j) {
    return (hash[j] - hash[i] + m) % m;
  }
}
\end{lstlisting}
\end{minipage}
};\node[fancytitle, right=10pt] at (box.north west) {\footnotesize Rolling Hashing};
\end{tikzpicture}
\end{multicols*}
\end{document}
